\chapter{Literature Survey}
\section*{Survey of Existing System}
The team researched for any automated software which was developed with the sole aim of detecting the Wolff Parkinson White Syndrome. The team was, however, not able to find any relevant system, which could be handling the problem statement. Through extensive research and reviewing of case studies, the team determined that no such software system had yet been developed or deployed. Henceforth, the team can confidently assert that the system undertaken for Wolff Parkinson White is indeed a fresh foray with no previous iterations or versions. The existing system or procedure in place for the detection of the Wolff Parkinson White Syndrome is of a traditional nature with little or no automation of procedures. The current process involves the evaluation of the ECGs by medical professionals after the occurrence of mild or fatal symptoms which is not a prudent measure to be followed. This syndrome is rare and may prove to be hard to be detected by medical professionals. This condition is genetic and inheritable at birth which may lead to increase in probability considering the syndrome is not detected or missed. The current procedure for detecting this syndrome goes as follows: 

\begin{enumerate}
	\item The patient may experience some symptoms, mild or serious and may deliberate whether to consult medical mediation. 
	\item The medical professional may be aware or unaware regarding the presence of this rare syndrome. 
	\item The medical professional may opt for multiple medical diagnosis which may lead to overburdening and lead to missing out on the syndrome's presence.
	\item The medical professional may fail to register the results or fail to provide the medical treatments for this curable syndrome. 
	\item The medical professional also may fail to register the patient's genetic history for further referrals of inherited instances of the Wolff Parkinson White Syndrome.  
\end{enumerate}

Research papers designed in accordance with deployment of Machine Learning techniques as well as from a medical science approach have been studied, observed and analyzed. Following are the inferences made:  

\begin{itemize}
	\item A method for automated generation of isochrone maps using electromechanical wave imaging (EWI) and machine learning techniques. EWI is a medical imaging modality that uses high-frequency ultrasound to measure the electromechanical wave propagation in biological tissues. By analyzing the time-delay between electrical activation and mechanical deformation, EWI can generate isochrone maps that provide information on the electrical activation patterns in the heart. The proposed method involves using a convolutional neural network (CNN) to automatically detect and segment the heart in EWI images, followed by computing the time-delay between electrical and mechanical signals to generate isochrone maps. The CNN is trained on a large dataset of EWI images with corresponding ground truth segmentation and isochrone maps. Overall, the paper presents a novel approach for automated generation of isochrone maps using EWI and machine learning techniques, which could potentially improve the diagnosis and treatment of cardiac arrhythmias. They also compare the results with traditional methods and show that the proposed method outperforms them in terms of accuracy and efficiency. 
	
	\item Using speckle tracking imaging to gain insights into the cardiac dysfunction that occurs in children and young adults with Wolff-Parkinson-White (WPW) syndrome. The authors found that patients with WPW syndrome had significantly lower global longitudinal strain (GLS) and global circumferential strain (GCS) compared to healthy controls. GLS and GCS are measures of myocardial deformation, which are used to assess myocardial function. The paper describes the specific patterns of cardiac dysfunction observed in patients with WPW syndrome, including impaired left ventricular systolic function and abnormal left ventricular filling. The authors also found that the severity of cardiac dysfunction was related to the presence of ventricular pre-excitation, a key feature of WPW syndrome. The study further explores the potential of STI as a diagnostic tool for WPW syndrome, particularly in detecting subtle changes in cardiac function that may not be evident with conventional imaging techniques. The authors suggest that STI could be a valuable addition to the diagnostic and monitoring tools for WPW syndrome. Overall, the paper provides a deep insight into the cardiac dysfunction in children and young adults with WPW syndrome using STI. The study highlights the importance of assessing cardiac function in patients with WPW syndrome and emphasizes the potential of STI as a diagnostic and monitoring tool for this population. 
	
	\item A case report of a then 15-year-old patient with intermittent Wolff-Parkinson-White (WPW) syndrome diagnosed by a single beat on a 12-lead electrocardiogram (ECG). WPW syndrome is a type of cardiac arrhythmia caused by an extra electrical pathway in the heart, which can lead to rapid heart rate and potentially life-threatening arrhythmias. Diagnosis of WPW syndrome typically requires observation of a characteristic ECG pattern called a delta wave, which is present during sinus rhythm. The authors of this case report describe a patient who presented with symptoms of palpitations and light-headedness but did not have a delta wave on initial ECG. However, a repeat ECG performed during an episode of palpitations revealed a delta wave, confirming the diagnosis of WPW syndrome. The authors emphasize the importance of performing repeat ECGs in patients with suspected WPW syndrome, as the characteristic delta wave pattern may be intermittent and not present during sinus rhythm. They also note that single-beat diagnosis of WPW syndrome is rare but possible, and clinicians should have a high index of suspicion for the condition in patients with symptoms suggestive of arrhythmia. Overall, the case report highlights the importance of careful ECG interpretation and the potential for intermittent WPW syndrome to be missed on initial evaluation.
	 
	\item Investigating the accuracy of algorithms in predicting the location of accessory pathways in pediatric patients with Wolff-Parkinson-White (WPW) syndrome. WPW syndrome is a type of cardiac arrhythmia caused by an abnormal electrical pathway in the heart, which can lead to rapid heart rate and potentially life-threatening arrhythmias. Accessory pathways can be located in different regions of the heart and identifying their location is important for planning treatment. The authors of this study developed and tested algorithms to predict the location of accessory pathways in pediatric patients with WPW syndrome. The aim of this study was to assess the predictive accuracy of 12 published algorithms for accessory pathway localization in pediatric WPW patients by comparing 12 lead resting ECG tracings with ECG tracings showing full ventricular preexcitation. The algorithms were based on demographic and electrocardiographic parameters and were compared to the actual location of the accessory pathway as determined by electrophysiological study (EPS). 
	
	\item The study included 64 pediatric patients with WPW syndrome who underwent EPS to determine the location of the accessory pathway. The accuracy of the algorithms varied depending on the location of the accessory pathway, with better accuracy for pathways located in the left side of the heart. For 12 lead resting ECG tracings, the algorithms published by D’Avila, Boersma, and Xie were found most accurate to determine exact AP location (58\%, 54\%, and 54\%, respectively). For laterality, the highest accuracy was found for the prediction of right-sided APs (from 42 to 78\%; median 74\%). For left-sided APs the median accuracy was 73\% (from 53 to 80\%), and for septal APs, the median accuracy was 68\% (from 48 to 78\%). From ECG tracings with full preexcitation during EPS the algorithm published by Boersma [11] yielded highest accuracy rates for exact AP localization (55\%). For laterality, highest accuracy was found for prediction of right-sided APs (from 52 to 78\%; median73\%). For left sided APs median accuracy was 73\% (from 58 to 83\%), and for septal Aps median accuracy was 65\% (from 50 to 75\%) The authors suggest that the algorithms could potentially be used as a non-invasive method for predicting the location of accessory pathways in pediatric patients with WPW syndrome, reducing the need for invasive EPS in some cases. However, they note that further studies are needed to confirm the accuracy of the algorithms and validate their use in clinical practice. 
	
	\item The paper presents a case report of a patient with congenital absence of left atrial appendage (LAA) and type A Wolff-Parkinson-White (WPW) syndrome, diagnosed using multimodal imaging techniques. The LAA is a small pouch-like structure in the left atrium of the heart, which can be a source of blood clots and increase the risk of stroke in patients with certain cardiac conditions. Congenital absence of LAA is a rare condition that may be associated with other cardiac abnormalities. Type A WPW syndrome is characterized by an extra pathway that is located in the anterior septal region of the heart. The authors of this case report describe a patient who presented with palpitations and was found to have type A WPW syndrome on electrocardiogram (ECG). Multimodal imaging techniques, including transthoracic echocardiography, cardiac magnetic resonance imaging (MRI), and computed tomography (CT) angiography, were used to further evaluate the patient's cardiac anatomy and function. The multimodal imaging studies revealed that the patient had congenital absence of LAA, which was associated with a smaller left atrium and increased blood flow velocities in the pulmonary veins. The authors thus emphasize the importance of using multimodal imaging techniques for the diagnosis and management of complex cardiac conditions, such as congenital absence of LAA and WPW syndrome. They note that further research is needed to better understand the pathophysiology and clinical implications of congenital absence of LAA and its association with other cardiac abnormalities. 
	
	\item The paper proposes a novel signal-adaptive multi-feature extraction algorithm for arrhythmia detection in electrocardiogram (ECG) signals. The algorithm aims to improve the accuracy of arrhythmia detection by utilizing multiple features of ECG signals and adapting to the signal characteristics. The proposed algorithm consists of four main steps: pre-processing, feature extraction, feature selection, and classification. In the pre-processing step, the ECG signal is filtered and segmented into individual heartbeats. Then, a set of features is extracted from each heartbeat using different signal processing techniques, including time-domain, frequency-domain, and wavelet-based methods. Next, a feature selection algorithm is employed to select the most relevant features for arrhythmia detection. The feature selection is performed based on a combination of statistical and correlation analysis to identify the most discriminative features. Finally, a machine learning classifier is trained using the selected features to classify each heartbeat as either a normal beat or an arrhythmic beat. The proposed algorithm is evaluated using the MIT-BIH Arrhythmia database and achieves high accuracy, sensitivity, and specificity compared to existing algorithms. Overall, the proposed algorithm offers a promising approach for arrhythmia detection using ECG signals, by combining multiple features and adapting to the signal characteristics. 
	
	\item The paper presents a study on the use of 3D-rendered electromechanical wave imaging (EMWI) for the localization of accessory pathways in paediatric patients with Wolff-Parkinson-White (WPW) syndrome. The study aims to evaluate the accuracy and feasibility of EMWI for guiding catheter ablation procedures in paediatric patients with WPW syndrome. The study included paediatric patients with WPW syndrome who underwent catheter ablation procedures guided by EMWI. EMWI involves the integration of 3D electroanatomic mapping with real-time ultrasound imaging to create a 3D-rendered image of the heart's electromechanical activation patterns. The use of EMWI also resulted in shorter procedure times and fewer catheter ablations compared to traditional fluoroscopic guidance. The study concludes that EMWI is a promising imaging modality for the localization of accessory pathways in paediatric patients with WPW syndrome. The use of EMWI may improve the accuracy and safety of catheter ablation procedures in paediatric patients, reducing the risk of complications and radiation exposure. Overall, the study suggests that 3D-rendered EMWI is a promising technique for localizing APs in paediatric patients with WPW syndrome. It offers a non-invasive, radiation-free, and accurate approach for identifying the location of APs, which can improve the success rate and safety of the catheter ablation procedure. 
	
	
	\item The paper proposes a novel approach for analyzing accessory pathways (APs) in patients with Wolff-Parkinson-White (WPW) syndrome using a multimodal deep learning model. WPW syndrome is a cardiac condition characterized by the presence of an additional electrical pathway between the atria and ventricles, leading to arrhythmias. The proposed deep learning model combines multimodal data, including electrocardiogram (ECG), intracardiac electrogram (IEGM), and imaging data, to improve the accuracy of AP analysis. The model consists of three main components: a feature extraction module, a multimodal fusion module, and a classification module. The proposed approach offers a promising alternative for AP analysis in patients with WPW syndrome. By integrating data from multiple modalities, the model improves the accuracy of AP localization and can potentially reduce the need for invasive procedures such as catheter ablation. However, further studies are needed to validate the performance of the model in larger patient populations and across different healthcare settings.  In the feature extraction module, each modality of data is processed separately to extract relevant features. The multimodal fusion module then combines the features from each modality to create a single representation of the AP. Overall, the study suggests that the proposed multimodal deep learning model offers a promising approach for analyzing APs in patients with arrhythmias. The model's ability to combine information from multiple sources of data improves the accuracy of AP analysis and has the potential to improve the diagnosis and treatment of patients with arrhythmias. 
	
	\item The paper provides an overview of Wolff-Parkinson-White (WPW) syndrome, a cardiac condition characterized by the presence of an accessory pathway in the heart that can cause rapid heartbeats or arrhythmias. The article discusses the pathophysiology of the condition, the clinical features, and the diagnostic and treatment options. The authors describe the pathophysiology of WPW syndrome, which involves the presence of an accessory pathway that connects the atria and ventricles of the heart, bypassing the normal electrical conduction system. The presence of this accessory pathway can cause electrical impulses to travel abnormally through the heart, leading to tachycardia or other arrhythmias. The article also discusses the clinical features of WPW syndrome, which can include palpitations, chest discomfort, shortness of breath, and syncope. The authors emphasize the importance of early diagnosis and treatment, as WPW syndrome can lead to serious complications, such as ventricular fibrillation and sudden cardiac death. The authors discuss the use of ECG in detecting the characteristic delta wave pattern, which is indicative of WPW syndrome. They also describe electrophysiology studies, which can be used to identify the location of the accessory pathway and guide catheter ablation therapy. The article concludes with a discussion of the treatment options for WPW syndrome, which include medications, catheter ablation, and surgical interventions. The authors emphasize that catheter ablation has become the preferred treatment option for WPW syndrome, as it offers a high success rate and low risk of complications. The table below provides the user with a quick, yet insightful glance of the characteristics of research papers utilized. 
\end{itemize}

\section*{Limitations of Existing System}

\subsection*{Gist of survey analysis:}   

In brief, there has been a comprehensive analysis of existing methodologies that have been deployed to increase the accuracy and frequency of prediction of the WPW syndrome. The literature survey consists of proposed methodologies such as automated generation of isochrone maps using electromechanical wave imaging (EWI), using speckle tracking imaging to gain insights into the cardiac dysfunction that occurs in children and young adults with Wolff-Parkinson-White (WPW) syndrome, diagnosis by a single beat on a 12-lead electrocardiogram (ECG) highlighting the importance of careful ECG interpretation and the potential for intermittent WPW syndrome to be missed on initial evaluation. These papers have also deployed use of 3D-rendered electromechanical wave imaging (EMWI) for the localization of accessory pathways in paediatric patients and early detection using a multimodal deep learning model. All observed systems have successfully achieved respective objectives. 

\subsection*{How far is the problem from being made into a solved complete problem:}   


As of today, not many systems exist for the sole purpose of detecting Wolff Parkinson White syndrome detection and mostly consist of research-based solutions for the detection problem. The research done for this syndrome is, most probabilistically suggesting the inclination towards medical basis for the subject matter at hand. Moreover, the research undertaken by the team also indicates that considering the very rare instances of any such systems in existence, many do not even come close to solving the accuracy issue of the detection problem with accuracy ranging over 70-80\% suggesting the instances of misdiagnosis or missing detections for the remaining 30\% of the overall cases. Another problem for such rare systems is the presence of issues such as false positives and false negatives which plague the system and lead to decrease in the evaluation parameters thus diminishing the system’s optimality and performance standards. 

Additionally, WPW detection is not officially standardised, with no calculated measures with for properly detection parameters. All these issues have been researched by the team and have tried to define the improvement possibilities for designing and developing a model addressing all these issues as well as improving the standard for the detection of Wolff Parkinson White detection. 

\subsection*{Observations of the analysis of all above points and improvement possibilities:}  

As stated, the Wolff Parkinson White is a rare syndrome, mentioned less frequently even in the medical contexts and discourses leading to less prevalence and information regarding the syndrome regardless of its severity or fatality probabilistic chances. Also, the issues associated with the detection subject of the Wolff Parkinson White syndrome such as, the detection’s accuracy and precision as well as dealing with the cases of false positives and false negatives associated with detection. All of these issues as listed can be used to demonstrate the improvement probabilities for the current limitations and drawbacks. The improvement possibility of increasing the accuracy along with other evaluation parameters such as precision and recall for the rare systems currently in place as well as solving the major issues of false positives and false negatives demonstrate the capability of this project’s undertaking to be a worthwhile endeavour. 

\subsection*{Observations on the technologies and methodologies that you feel are performing better: }  

In the current time frame of developing this report, rare instances of technologies or different methodologies-based models were discovered. Based on the research undertaken, it can be stated that methodologies such as multimodal imaging and speckle track imaging can be stated to be performing better among rare instances of detection implementations with accuracy parameters averaging around 75\%. Also, the use of machine learning classification models such as Random Forest Method, Support Vector Machine, etc. are developed for the textual data derived from ECG readings. Altogether, based on the research and evaluation performed by our time, some methodologies can be stated to be performing better, however with no public releases of web based or application user interfaces for utilizing the detection features of any such methodology in real time. 


