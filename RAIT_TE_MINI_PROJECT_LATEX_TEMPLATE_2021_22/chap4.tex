\chapter{Results and Discussion}

This chapter presents the results generated. Compare them w.r.t. the existing solutions discussed in the literature survey. Add your project outcomes (screenshots of implementation). This is the brainstorming part. Understand, analyse, visualise why the results are the way they are. 

\section{Implementation Details}


\section{Result Analysis}
The result of this web application detection system primarily focuses on the precise detection of whether or not the patient has WPW syndrome. The accuracy for the same is 96.06\% and thus the model can be considered a reliable source of detection. In addition, the system does not just accurately depict the ECG analysis outcome but also allows users to check if an ECG is recommended or not through the pre-listed common WPW symptoms page which makes use of a probability-based approach. This has been done in order to make the system advise ECG only to those patients who have an urgent need to do so. Apart from this, the web application has the ability to connect patients with the nearest medical centers/ hospitals offering treatment for WPW syndrome and a list of the treatment methodology and medicines advised in general. It is important to note that we do not recommend any of the treatment processes or medicine consumption. In the process, the system is effective in terms of acting as a one-stop solution for the patients. 
