\chapter{Introduction}
\section{Overview}\label{sec1}
%\subsection{AI \& ML}\label{subsec1}

%\subsubsection{Data Science}

WPW syndrome (Wolff-Parkinson-White syndrome) is a cardiac arrhythmia characterized by an abnormal pathway between the atria and ventricles of the heart, which can lead to a rapid and irregular heartbeat. The condition is usually diagnosed using an electrocardiogram (ECG), which measures the electrical activity of the heart. However, WPW syndrome is often associated with low detection rates which increases the criticality of this syndrome and emphasizing the importance of early detection. The low detection rate of WPW syndrome can be attributed to several factors. One of the primary reasons is that the condition is often asymptomatic, meaning that many people with WPW syndrome may not experience any noticeable symptoms. In some cases, WPW syndrome may only be discovered incidentally during routine medical tests such as an electrocardiogram (ECG).  

Another reason for the low detection rate of WPW syndrome is that it is a relatively rare condition. According to some estimates, WPW syndrome affects only around 0.1-0.3\% of the general population. As a result, many healthcare providers may not be familiar with the condition or may not consider it as a possible diagnosis in patients with symptoms that can be attributed to other causes. In addition to the listed parameters, WPW syndrome can be difficult to diagnose, especially in cases where the extra pathway is intermittent or only present during certain heart rhythms. In some cases, additional testing such as a Holter monitor or electrophysiological study may be necessary to detect the condition. 



Overall, the low detection rate of WPW syndrome underscores the importance of regular medical check-ups and screenings, especially for individuals with a family history of the condition or other risk factors. If a patient is experiencing symptoms such as rapid or irregular heartbeats, it is important to seek medical attention promptly to rule out any underlying cardiac conditions.  




\newpage

This gives an overview \cite{kunjir2017data} about chapter. The following Equation~\ref{eq2} about parameters.

 The following Figure~\ref{fig:rait-dypu-logo} is the DYPU Logo.

\begin{figure}[h]
	\centering
		\includegraphics{rait-dypu-logo.png}
	\caption{DY Patil Deemed University logo}
	\label{fig:rait-dypu-logo}
\end{figure}


%\begin{figure}[h]
%	\centering
%	\includegraphics[width=0.7\linewidth]{rait-dypu-logo}
%	\caption{Figure Caption Here}
%	\label{fig:plot}
%\end{figure}


\begin{equation} \label{eq1}
	\beta \leftarrow \frac{5 (\pi - 1)}{\lambda^2} 
\end{equation}




\cite{kunjir2017data}



\begin{enumerate}
	\item Which observations strongly motivated you to take up this problem domain to work.
	
	\item Which observations strongly motivated you to take up this problem domain to work.
\end{enumerate}

\begin{description}
	\item[AI] Which observations strongly motivated you to take up this problem domain to work.
\end{description}

\begin{center}	
	Which observations strongly motivated you to take up this problem domain to work.
\end{center}


The following Table~\ref{tab1} shows the Name of the Students.
\begin{table}[!ht] \label{tab1}
	\caption{Name of Students}
	\centering
	\begin{tabular}{|p{5cm}|p{5cm}|}
		\hline
		\textbf{Name}                  & \textbf{Roll No.} \\ \hline
		Mr. Kishor Deepak Waghe        & (20CE5009)        \\ \hline
		Mr. Pradyumna Sameer Mandawkar & (20CE5010)        \\ \hline
		Ms. Prabhuti Jayesh Patil      & (20CE5014)        \\ \hline
		Which observations strongly motivated you to take up this problem domain to work.      & (19CE1053)        \\ \hline
	\end{tabular}
\end{table}

\section{Motivation}
There has been a visible and drastic change towards a positive side when it comes to how the healthcare sector has adapted to the evolution of technology. The COVID-19 pandemic further accelerated the need for use of remote systems when dealing with patient of any kind. This is where the proposed system comes into the picture ensuring an easily available, cost and time sensitive solution for patients who may be suffering from the Wolff Parkinson White Syndrome. The motivation behind selection and development of this system is: 


\begin{itemize}
	\item Firstly, it could help in the early and accurate diagnosis of this very rare condition, which is important for effective treatment. 
	\item Secondly, it could improve the speed and efficiency of diagnosis, particularly in settings where access to medical professionals or resources is limited. 
	\item A machine learning model could potentially use a large dataset of ECG recordings to identify patterns or features that are associated with WPW syndrome. 
	\item The model could also be designed to provide a probability score or confidence level for its classification, allowing medical professionals to make more informed decisions about treatment. 
	\item However, it's important to note that developing a machine learning model for diagnosing WPW syndrome would require a large, diverse and representative dataset of ECG recordings, as well as appropriate regulatory approvals and ethical considerations.  
	\item Additionally, the machine learning model should be used in conjunction with trained medical professionals, as it is intended to assist with diagnosis and improve patient outcomes. 
\end{itemize}


\section{Problem Statement and Objectives}
The primary objectives of our web-application would be to provide:  
\begin{enumerate}
	\item Provides streamlined way for individuals to perform a quick check up. 
	\item The first part of our project is to create a separate model that will be used to convey if the patient needs to take an ECG test in the first place.
	\item The mentioned model is based on symptoms that a person with a heart problem and in need of check-up might suffer from. 
	\item The various symptoms are as follows: 
	\begin{itemize}
		\item High heart rate 
		\item Dizziness 
		\item Fainting 
		\item Constant Tiredness (Fatigue) 
		\item Anxiousness			 
		\item Chest Pain 
		\item Difficulty Breathing
	\end{itemize}
	\item The inputs the user will provide will be yes/no if they suffer from the above-mentioned symptoms. 
	\item Using the above inputs, the project will provide a prompt if they should undergo an ECG test. 
	\item The core element of our project is to develop a model such that by taking the ECG data input, can detect if the person is suffering from Wolff-Parkinson-White Syndrome. 
	\item The next part of the project is to provide a list of medical centers where the user can go for treatment if needed. 
	\item And another part would be to display the various available treatments they can take if they suffer from Wolff-Parkinson-White Syndrome.
	 
	
\end{enumerate}

\section{Organization of the report}

The report is organised as follows: The Chapter 2 reviews the literature. Chapter 3 focuses on defining the system's issue. That includes problem categorization, proposed technologies, device architecture, and hardware/software requirements. On the other hand, Chapter 5 describes the inference and future work on the technique to be utilized as a more improved model.
